\section{Community Fund}
\label{S:9}

The network runs a fully decentralised community fund to compensate contributors for their efforts. The 
fund is fully decentralised meaning any network participant may submit a proposal for a project and all 
projects are voted on democratically by the network, with all network participants (having staking weight) 
eligible to vote. Since NAVCoin does not have master nodes, there are no master nodes with centralised 
voting power or control over community funds. The community fund is implemented on a protocol level and 
there are no nodes or addresses holding the funds.

Thus, the nature and scope of projects funded by the community fund is decided by and voted on by all community 
members. This decentralises conception and execution of projects and ideally, this incentivises the creation 
of projects which are in the interests of the network. 

\subsection{Funds Available}

Currently 0.5 NAV per block is generated into the community fund which results in approximately 525,600 NAV
\footnote{Assuming 2 blocks per minute x 60 minutes per hour x 24 hours per day x 365 days per year} 
generated per annum, which is available as payouts for community projects. In addition, network participants 
may donate funds to the community fund.

\subsection{Dual-vote Consensus and Procedure}

The fund is operated under a dual voting mechanism, meaning the network will vote on each project in two
stages:
\begin{itemize}
\item Proposal stage: Firstly, the network will vote on their acceptance of the project nature and scope.
\item Payment request stage: Secondly, the network will vote on whether to release funds to the benefactor 
depending on whether progress on the project has been completed satisfactorily.
\end{itemize}

Thus for a network participant to be paid out from the community fund, they must first submit a proposal 
outlining the nature, scope, timeframe and total cost of their project.  If their proposal is accepted, 
they would be expected to carry out work on their project and then submit a payment request for the second 
stage vote and only when the second stage is voted on successfully would they be paid. One proposal could 
have multiple payment requests summing up to no more than the total amount outlined in the proposal.

At each stage, network participants vote on proposals and payment requests by selecting their vote in their 
staking wallet, and if they stake a block, their vote is recorded on the blockchain. Thus the maximum number 
of votes recorded for each voting period is equal to the number of blocks and every stake corresponds to one vote. 

Voters can choose to cast a yes vote, no vote or abstain for each proposal and payment request. If the number 
of yes votes meets the required acceptance threshold and voting quorum, the proposal or payment request is 
accepted at the end of that voting period. If the number of no votes meets the required rejection threshold 
and voting quorum, the proposal or payment request is rejected at the end of that voting period and can no 
longer be voted on. Proposals and payment requests which do not meet these thresholds remain pending and 
can be voted on in the next period, up to a maximum of 6 and 8 periods for proposals and payment requests 
respectively, after which they will expire.

\subsection{Technical Specifications and Voting Thresholds}

The community fund uses new opcodes to handle creation of proposals, payment requests, voting and donations. Technical 
details of the community fund implementation can be found 
\href{https://github.com/aguycalled/cfundpaper/blob/master/cfund.pdf}{here}. The current specifications and 
voting thresholds of the community fund are as follows:
\begin{table}[h]
\centering
\begin{tabular}{l l}
\hline
Community fund contribution & 0.5 NAV per block \\
Voting period length & 20,160 blocks\\
Proposal acceptance threshold & 70\% of total votes \\
Proposal rejection threshold & 70\% of total votes \\
Proposal voting quorum & 50\% \\
Maximum voting periods for deliberation on proposals & 6 voting periods \\
Payment request acceptance threshold & 70\% of total votes \\
Payment request rejection threshold & 70\% of total votes \\
Payment request voting quorum & 50\% \\
Maximum voting periods for deliberation on payment requests & 8 voting periods \\
Proposal submission fee & 50 NAV (donated to the community fund) \\
\hline
\end{tabular}
\caption{Technical specifications of the Community Fund}
\end{table}
