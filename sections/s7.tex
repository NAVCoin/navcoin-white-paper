\section{Incentive / Inflation Schema}
\label{S:7}
\subsection{Incentive Mechanism}

Each block contains a special transaction known as the coinstake that generates new NAV for the 
creator of the block. At the time of writing, the reward amount generated is fixed at 2 NAV per block. 

Similar to Bitcoin, an incentive can also be provided via transaction fees. If the output value 
of a transaction is less than its input value, the difference is a transaction fee that is added 
to the incentive value of the block containing the transaction. 

Since nodes can only receive these rewards if they are online and staking, this incentivises nodes to 
(i) create blocks for recording transactions and (ii) stay online to support the network. The latter 
is particularly important for network security because the more participants who try to receive 
rewards by having coins online and staking, the more participants there are to determine consensus on 
the state of the blockchain. More participants online makes the network more resistant to bad actors trying 
to impose their own version of the state of the blockchain through 51\% attacks on the network.

\subsection{Inflation}

The fixed reward amount, combined with the community fund contribution of 0.5 NAV per block, results in 
a net increase in supply of 2.5 NAV for every block created. Assuming there are 1,051,200 blocks created 
per annum (calculated by 2 blocks per minute x 60 x 24 x 365), this results in an increase of 2,628,000 
NAV per annum in money supply available to the network. At the time of writing, the total money supply of 
NAV available to the network is approximately 63,995,000 NAV, which results in an inflation rate of 
approximately 4.11\%. However, given that the increase in money supply per annum is fixed and the total 
supply is increasing, the percentage inflation rate will fall over time.


